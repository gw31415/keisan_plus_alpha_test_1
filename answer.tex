\documentclass[uplatex,b5paper]{jsarticle}
\usepackage{bm}
\usepackage{empheq}
\usepackage{amsmath}
\usepackage{amssymb}
\usepackage{bxpapersize}
\usepackage{kyotoexmath}

\setfootmark{天輝主/Gω3}
\title{第 n 回}
\subject{計 算 \(+\alpha\) テ ス ト \\\vspace{.5cm} 略 解}
\setprereadme{デリンジャ-なら満点取れるテスト}
\setscoresum6
\setreadme{
  \begin{enumerate}
    \item 電卓等の使用は禁止です.
    \item 途中経過は採点対象にしません.
    \item 今回の制限時間は20分です.
    \item 問題数は3問です.
    \item 計算テストと捉えるには少しマイナーな問題となっております.
    \item 終了後に, 私のDMに解答を送ってください. 訂正は受け付けません. 即座に採点し, 平均点と成績優秀者を発表します.
    \item 今回の作問者は{\it @天輝主\#3141}, アドバイザーは{\it @らむぞー\#5342},{\it @雪葉\#3875}です.
    \item 本家計算テストにて作問者募集されてます. 有志の方々は{\it @mameshiba!\#2047}さんにお声がけください.
  \end{enumerate}
}
\begin{document}
\maketitle

\begin{prob}3
  次の漸化式で定義された数列 \(\{a_n\}\) の一般項を求めよ.
  \[a_{n+2} - 2a_{n+1} + a_n = n\cdot2^n - {}_n\mathrm C_4\]
  ただし \(a_1 = a_2 = 0\) であり, \(n<m\)のときの \({}_n\mathrm C_m = 0\) と定義する.

  \vspace{2cm}
  まず \(n\geq2\) についての一般項を考える.

  以下の恒等式を変形していく.
  \begin{eqnarray*}
    1+x+x^2+\cdots+x^n &=& \frac{x^{n+1}-1}{x-1}\\
    x+2x^2+3x^3+\cdots+nx^n &=& x\cdot\frac d{dx}\left(\frac{x^{n+1}-1}{x-1}\right)
  \end{eqnarray*}
  よって \(x=2\) として計算して,
  \[\sum_{k=1}^{n-1}k\cdot2^k=(n-2)\cdot2^n+2\]
  がわかる.

  さて, 漸化式より,
  \begin{eqnarray*}
    (a_{n+2}-a_{n+1})-(a_{n+1}-a_n) &=& n\cdot2^n-{}_n\mathrm C_4\\
    \sum_{k=1}^{n-1}\left((a_{k+2}-a_{k+1})-(a_{k+1}-a_k)\right) &=& \sum_{k=1}^{n-1}\left(k\cdot2^k-{}_k\mathrm C_4\right)\\
    a_{n+1}-a_n &=& (n-2)\cdot2^n+2-{}_n\mathrm C_5\\
    \sum_{k=1}^{n-1}(a_{k+1}-a_k) &=& \sum_{k=1}^{n-1}\left(k\cdot2^k-2^{k+1}+2-{}_k\mathrm C_5\right)\\
    a_n &=& (n-2)\cdot2^n+2-4(2^n-1)+2(n-1)-{}_n\mathrm C_6\\
        &=& (n-4)\cdot2^n+2n+4-{}_n\mathrm C_6\\
  \end{eqnarray*}
   \[\Bigl(= (n-4)\cdot2^n-\frac{n^7}{720}+\frac{7n^6}{240}-\frac{35n^5}{144}+\frac{49n^4}{48}-\frac{203n^3}{90}+\frac{49n^2}{20}+n+4\Bigr)\]
  \[\Bigl(= (n-4)\cdot2^n-\frac1{720}(n^7-21n^6+175n^5-735n^4+1624n^3-1764n^2-720n-2880)\Bigr)\]

  \(n=1\) でもこれでよい.
\end{prob}
\newpage

\begin{prob}2
  \(xyz\)直交座標系において,
  \[(3,1,4), (1,5,9), (2,6,5), (3,5,8)\]
  を4頂点とする四面体の体積を求めよ.

  \vspace{2cm}
  四面体をつくる3ベクトルは以下.
  \[(0,4,4), (2,0,-1), (1,-1,3)\]
  ただし四面体は等積変形した.
  それぞれ \(\bm A,\bm B,\bm C\) とする.

  \(\bm A,\bm C\) がつくる三角形を \(S\) とすると,
  大きさが \(S\) の面積に等しく
  \(S\) に垂直なベクトル \(S\bm n\) は,
  (\(\bm n\)を単位ベクトルとする)
  \[S\bm n=\frac12
    \left(
      \begin{array}{ccc}
        4\\-8\\8
      \end{array}
    \right)
    =
    \left(
      \begin{array}{ccc}
        2\\-4\\4
      \end{array}
    \right)
  \]
  \[
    \left(
      \begin{cases}
        |S\bm n|=\frac12\sqrt{|\bm B|^2|\bm C|^2-(\bm B\cdot\bm C)^2}\\
        S\bm n\cdot\bm B=S\bm n\cdot\bm C=0
      \end{cases}
      \text{は簡単な計算によって確かめられる}
    \right)
  \]
  ( \(S\bm n\) は計算用紙では外積によって計算する )

  \(S\) を底面とした高さを\(h\)とすると,
  \[h=|(1,-1,3)\cdot \bm n|\]

  よって求める体積 \(V\) は
  \[V=\frac13Sh
    = \frac13S\cdot\left|
    \left(
      \begin{array}{ccc}
        1\\-1\\3
      \end{array}
    \right)
      \cdot\bm n\right|
    = \frac13\left|
    \left(
      \begin{array}{ccc}
        1\\-1\\3
      \end{array}
    \right)
      \cdot S\bm n\right|
    = \frac13\left|
    \left(
      \begin{array}{ccc}
        1\\-1\\3
      \end{array}
    \right)
    \cdot
    \left(
      \begin{array}{ccc}
        2\\-4\\4
      \end{array}
    \right)\right|
    =6
  \]
\end{prob}
\newpage

\begin{prob}1
  三次方程式
  \[2x^3+3x^2-6x-k=0\]
  の異なる実数解の個数を求めよ.

  \vspace{2cm}
  \[f: x \mapsto 2x^3+3x^2-6x\]
  極大値と極小値を取る \(x\) をそれぞれ \(s,t\quad(s<t)\) とする.
  \[f'(x)=6(x^2+x-1)\]
  であるが, \(s,t\) は \(f'(s)=f'(t)=0\) を満足する.

  三次関数の点対称性より,
  極大点と極小点の中点は与式の曲線を通る.
  \begin{eqnarray*}
    f(s)+f(t) &=& 2\cdot f\left(\frac{s+t}2\right)\\
              &=& 2\cdot f\left(-\frac12\right)\\
              &=& 7
  \end{eqnarray*}

  \begin{eqnarray*}
    f(s)-f(t) &=& \int_t^sf'(x)dx\\
              &=& \int_t^s6(x^2+x-1)dx\\
              &=& 6\cdot\int_t^s(x-s)(x-t)dx\\
              &=& 6\cdot\frac16(t-s)^3\\
              &=& (t-s)^3\\
              &=& \left(\sqrt5\right)^3\\
              &=& 5\sqrt5
  \end{eqnarray*}

  連立して,
  \[\text{極大値は}\ \frac{7+5\sqrt5}2,\quad\text{極小値は}\ \frac{7-5\sqrt5}2\]

  以上より
  \begin{eqnarray*}
      \begin{cases}
        \text{1つ} &\left(\displaystyle k<\frac{7-5\sqrt5}2\ \text{または}\ k>\frac{7+5\sqrt5}2 \right)\\
        \text{2つ} &\left(\displaystyle k=\frac{7-5\sqrt5}2\ \text{または}\ k=\frac{7+5\sqrt5}2 \right)\\
        \text{3つ} & \left(\displaystyle \frac{7-5\sqrt5}2<k<\frac{7+5\sqrt5}2 \right)
      \end{cases}
  \end{eqnarray*}

\end{prob}
\end{document}
