\documentclass[uplatex,b5paper]{jsarticle}
\usepackage{empheq}
\usepackage{amsmath}
\usepackage{amssymb}
\usepackage{bxpapersize}
\usepackage{kyotoexmath}

\setfootmark{天輝主/Gω3}
\title{第 n 回}
\subject{計 算 \(+\alpha\) テ ス ト}
\setprereadme{デリンジャ-なら満点取れるテスト}
\setscoresum{6}
\setreadme{
  \begin{enumerate}
    \item 電卓等の使用は禁止です.
    \item 途中経過は採点対象にしません.
    \item 今回の制限時間は20分です.
    \item 問題数は3問です.
    \item 計算テストと捉えるには少しマイナーな問題となっております.
    \item 終了後に, 私のDMに解答を送ってください. 訂正は受け付けません. 即座に採点し, 平均点と成績優秀者を発表します.
    \item 今回の作問者は{\it @天輝主\#3141}, アドバイザーは{\it @らむぞー\#5342},{\it @雪葉\#3875}です.
    \item 本家計算テストにて作問者募集されてます. 有志の方々は{\it @mameshiba!\#2047}さんにお声がけください.
  \end{enumerate}
}
\begin{document}
\maketitle

\begin{prob}3
  次の漸化式で定義された数列 \(\{a_n\}\) の一般項を求めよ.
  \[a_{n+2} - 2a_{n+1} + a_n = n\cdot2^n - {}_n\mathrm C_4\]
  ただし \(a_1 = a_2 = 0\) であり, \(n<m\)のときの \({}_n\mathrm C_m = 0\) と定義する.
\end{prob}
\begin{prob}2
  \(xyz\)直交座標系において,
  \[(3,1,4), (1,5,9), (2,6,5), (3,5,8)\]
  を4頂点とする四面体の体積を求めよ.
\end{prob}
\begin{prob}1
  三次方程式
  \[2x^3+3x^2-6x-k=0\]
  の異なる実数解の個数を求めよ.
\end{prob}
\probends
\end{document}
